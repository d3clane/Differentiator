Ребят, вы че издеваетесь?Я понимаю, что вам хочется просто расслабиться и наслаждаться жизнью.И не думать о дифференцировании, решении уравнений.У меня просто завален весь direct"Арман, ты же умеешь дифференцировать, продифференцируй, тебе жалко что ли?"Мне не сложно, но я не могу дифференцировать просто так! Поэтому давайте поступим так.Целый год мои дифференциалы были платными.Для того, чтобы получить дифференцирование, нужно было заплатить.Сегодня мне захотелось, чтобы через мой продукт смог пройти каждый.Чтобы у каждого была возможность не отчислиться.Потому что не каждый может позволить себе дифференциал, когда в приоритете по расходам сначала идёт семья/кредиты/ипотеки.Не упусти свой шанс! Бесплатное дифференцирование: 
\begin{gather}
\end{gather}
\begin{math}
2.00 ^ {\sin x }\\
\end{math}
Ящеры нападают, нужно срочно взять эту производную ради победы
\begin{gather}
\end{gather}
\begin{math}
2.00 ^ {\sin x }\\
\end{math}
Возьмем производную от: 
\begin{gather}
\end{gather}
\begin{math}
x \\
\end{math}
Несложно заметить это преобразование
\begin{gather}
\end{gather}
\begin{math}
1.00 \\
\end{math}
Возьмем производную от: 
\begin{gather}
\end{gather}
\begin{math}
\sin x \\
\end{math}
Несложно заметить это преобразование
\begin{gather}
\end{gather}
\begin{math}
\cos x \cdot 1.00 \\
\end{math}
Возьмем производную от: 
\begin{gather}
\end{gather}
\begin{math}
2.00 ^ {\sin x }\\
\end{math}
Любопытный читатель может показать этот переход самостоятельно, 
\begin{gather}
\end{gather}
\begin{math}
2.00 ^ {\sin x }\cdot \ln 2.00 \cdot \cos x \cdot 1.00 \\
\end{math}
Выражение после взятия производной:

\begin{gather}
\end{gather}
\begin{math}
2.00 ^ {\sin x }\cdot \ln 2.00 \cdot \cos x \cdot 1.00 \\
\end{math}
***** не понятно, но очень интересно. Сделаем выражение более понятным
Упростим это выражение: 
\begin{gather}
\end{gather}
\begin{math}
\ln 2.00 \\
\end{math}
Любопытный читатель может показать этот переход самостоятельно, 
\begin{gather}
\end{gather}
\begin{math}
0.69 \\
\end{math}
Даже не буду ничего говорить, 
\begin{gather}
\end{gather}
\begin{math}
\cos x \cdot 1.00 \\
\end{math}
Очевидно, что это преобразование верно
\begin{gather}
\end{gather}
\begin{math}
\cos x \\
\end{math}
Итоговое выражение после упрощений:
\begin{gather}
\end{gather}
\begin{math}
2.00 ^ {\sin x }\cdot 0.69 \cdot \cos x \\
\end{math}
Итоговый ответ: 
\begin{gather}
\end{gather}
\begin{math}
2.00 ^ {\sin x }\cdot 0.69 \cdot \cos x \\
\end{math}
Разложение по маклорену: 
\begin{gather}
\end{gather}
\begin{math}
1.00 + 0.69 \cdot x + 0.48 \cdot x ^ {2.00 }+ -0.36 \cdot x ^ {3.00 }+ -1.69 \cdot x ^ {4.00 }+ -2.48 \cdot x ^ {5.00 }+ 3.18 \cdot x ^ {6.00 }+ 24.09 \cdot x ^ {7.00 }+ 40.65 \cdot x ^ {8.00 }+ -124.24 \cdot x ^ {9.00 }+ -878.27 \cdot x ^ {10.00 }\\
\end{math}
График функции:

\begin{figure}[H]
\centering
\includegraphics[scale=0.6]{Graphs/graph_1337_time_23:10:49.png}
\end{figure}
График разложение по маклорену:

\begin{figure}[H]
\centering
\includegraphics[scale=0.6]{Graphs/graph_1338_time_23:10:49.png}
\end{figure}
Сравнение графиков функции и маклорена в окрестности нуля:

\begin{figure}[H]
\centering
\includegraphics[scale=0.6]{Graphs/graph_1339_time_23:10:49.png}
\end{figure}
График разницы между функцией и разложение по маклорену:

\begin{figure}[H]
\centering
\includegraphics[scale=0.6]{Graphs/graph_1340_time_23:10:49.png}
\end{figure}
