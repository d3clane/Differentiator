\documentclass[12pt,a4paper]{extreport}
\usepackage[l2tabu,orthodox]{nag}

\usepackage{indentfirst}
\usepackage[labelsep=period]{caption}
\usepackage{amssymb,amsmath,amsthm}
\usepackage{fontspec}
\usepackage{float}

\setmainfont[Ligatures=TeX]{STIX}
\newfontfamily{\cyrillicfont}[Ligatures=TeX]{STIX}
\setmonofont{Fira Mono}
\newfontfamily{\cyrillicfonttt}{Fira Mono}

\usepackage{polyglossia}
\setdefaultlanguage{russian}
\setotherlanguage{english}

\usepackage{subcaption}
\usepackage{graphicx}
\graphicspath{img/}
\DeclareGraphicsExtensions{.pdf,.png,.jpg}

\usepackage{color}
\definecolor{darkblue}{rgb}{0,0,.75}
\definecolor{darkred}{rgb}{.7,0,0}
\definecolor{darkgreen}{rgb}{0,.7,0}

\usepackage[normalem]{ulem}
\setlength{\marginparwidth}{2cm}
\usepackage[textwidth=4cm,textsize=tiny]{todonotes}
\newcommand{\fix}[2]{{\textcolor{red}{\uwave{#1}}\todo[fancyline]{#2}}}
\newcommand{\hl}[1]{{\textcolor{red}{#1}}}
\newcommand{\cmd}[1]{{\ttfamily{\textbackslash #1}}}

\newcommand{\vrb}[1]{\PVerb{#1}}
\newcommand{\vrbb}[1]{\texttt{\textbackslash}\PVerb{#1}}

\usepackage[
    draft = false,
    unicode = true,
    colorlinks = true,
    allcolors = blue,
    hyperfootnotes = true
]{hyperref}

\usepackage{tikz}
\usetikzlibrary{graphs, quotes}
\usepackage{relsize}
\usepackage{ytableau}

%\usepackage{titlesec}
%\titleformat{\subsection}{\normalfont\large\bfseries}{\thesubsection.}{\smallskip}{}
\renewcommand \thesection{\Roman{section}}
\renewcommand \thesubsection{\arabic{subsection}}

\theoremstyle{plain}
\newtheorem{theorem}{Теорема}
\newtheorem{lemma}{Лемма}
\newtheorem{proposition}{Утверждение}
\newtheorem{corollary}{Следствие}
\theoremstyle{definition}
\newtheorem{definition}{Определение}
\newtheorem{notation}{Обозначение}
\newtheorem{example}{Пример}

\newcounter{homework}
\setcounter{homework}{0}
\newenvironment{homework}[2][\thehomework+1]%
{\setcounter{homework}{#1}\addtocounter{homework}{-1}\refstepcounter{homework}
\setcounter{subsection}{\thehomework-1}%
\subsection{#2}}%
{\newpage}
\newenvironment{homework*}[1]%
{\subsection*{#1}}%
{\newpage}

\newcounter{task}
\counterwithin{task}{homework}
\newenvironment{task}[1]%
{\setcounter{task}{#1}\addtocounter{task}{-1}\refstepcounter{task}%
\par\noindent\textbf{Задача~\thetask. }}%
{\smallskip}

\newenvironment{solution}%
{\par\noindent\textbf{Решение. }}%
{\bigskip}

\newcommand\abs[1]{\left\lvert #1 \right\rvert}
\newcommand\ceil[1]{\left\lceil{#1}\right\rceil}
\newcommand\floor[1]{\left\lfloor{#1}\right\rfloor}
\newcommand{\divby}{\;\raisebox{-0.4ex}{\vdots}\;} 


\title{<<Математические основания алгоритмов и сложность вычислений>>}
\author{Овакимян Арман Б05-332}

\begin{document}
\maketitle
\tableofcontents

\begin{solution}

Автор данной статьи решил одну из важнейших задач последнего тысячелетия. В связи с этим он хотел нанять профессионального переводчика, который сделал бы эту статью подходящей для публикации в популярном научном журнале. Автор надеялся заработать много денег, чтобы прожить жизнь безбедно, но на данный момент у него нет денег даже на переводчика, поэтому была совершенна отчаянная попытка перевести самому. Автор надеется, что мировое сообщество математиков примет данную статью, поймет ее и заплатит ему много денег.

Отдельные благодарности от автора за помощь в создании статьи 
\begin{enumerate}
    \item Перуну - великому богу-громовержцу                                                                   \
    \item Байкальской водице, выпив которую автор преисполнился в своем познании математического анализа.      \
    \item Лечащему врачу в психиатрическое больнице №7, который ухаживал за автором во время написания статьи. \
\end{enumerate}

Перейдем к самой статье. \\
according to legend, the ancient Rus were able to defeat the lizardsby taking this derivative:
\begin{gather}
\end{gather}
\begin{math}
2.00 ^ {\sin x }\\
\end{math}
Let's take the derivative of: 
\begin{gather}
\end{gather}
\begin{math}
x \\
\end{math}
Nikto ne zametit, chto ya ne smog perevesti eto dlya svoe' stat'i: 
\begin{gather}
\end{gather}
\begin{math}
1.00 \\
\end{math}
Let's take the derivative of: 
\begin{gather}
\end{gather}
\begin{math}
\sin x \\
\end{math}
Don't ask me to prove this: 
\begin{gather}
\end{gather}
\begin{math}
\cos x \cdot 1.00 \\
\end{math}
Let's take the derivative of: 
\begin{gather}
\end{gather}
\begin{math}
2.00 ^ {\sin x }\\
\end{math}
Perun sent me the solution and I don't have no right to not to believe: 
\begin{gather}
\end{gather}
\begin{math}
2.00 ^ {\sin x }\cdot \ln 2.00 \cdot \cos x \cdot 1.00 \\
\end{math}
the ancient Rus, like us, got this result

\begin{gather}
\end{gather}
\begin{math}
2.00 ^ {\sin x }\cdot \ln 2.00 \cdot \cos x \cdot 1.00 \\
\end{math}
no one gives a **** what's going on here, but according to the standards I have to say it - "Ya sobirayus uprostit virazhenie))))".Let's simplify this expression: 
\begin{gather}
\end{gather}
\begin{math}
\ln 2.00 \\
\end{math}
Even a monkey can learn to differentiate, why won't you do it by yourself?
\begin{gather}
\end{gather}
\begin{math}
0.69 \\
\end{math}
Slozhno ne ponyat, сhto I mean: 
\begin{gather}
\end{gather}
\begin{math}
\cos x \cdot 1.00 \\
\end{math}
I have a proof of this theorem, but there is not enough space in this margin: 
\begin{gather}
\end{gather}
\begin{math}
\cos x \\
\end{math}
Final expression after simplifications:
\begin{gather}
\end{gather}
\begin{math}
2.00 ^ {\sin x }\cdot 0.69 \cdot \cos x \\
\end{math}
Итоговый ответ: 
\begin{gather}
\end{gather}
\begin{math}
2.00 ^ {\sin x }\cdot 0.69 \cdot \cos x \\
\end{math}
Разложение по маклорену: 
\begin{gather}
\end{gather}
\begin{math}
1.00 + 0.69 \cdot x + 0.48 \cdot x ^ {2.00 }+ -0.36 \cdot x ^ {3.00 }+ -1.69 \cdot x ^ {4.00 }+ -2.48 \cdot x ^ {5.00 }+ 3.18 \cdot x ^ {6.00 }+ 24.09 \cdot x ^ {7.00 }+ 40.65 \cdot x ^ {8.00 }+ -124.24 \cdot x ^ {9.00 }+ -878.27 \cdot x ^ {10.00 }\\
\end{math}
График функции:

\begin{figure}[H]
\centering
\includegraphics[scale=0.6]{Graphs/graph_1337_time_02:20:09.png}
\end{figure}
График разложение по маклорену:

\begin{figure}[H]
\centering
\includegraphics[scale=0.6]{Graphs/graph_1338_time_02:20:09.png}
\end{figure}
Сравнение графиков функции и маклорена в окрестности нуля:

\begin{figure}[H]
\centering
\includegraphics[scale=0.6]{Graphs/graph_1339_time_02:20:09.png}
\end{figure}
График разницы между функцией и разложение по маклорену:

\begin{figure}[H]
\centering
\includegraphics[scale=0.6]{Graphs/graph_1340_time_02:20:09.png}
\end{figure}
\end{solution}

\end{document}